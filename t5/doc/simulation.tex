\section{Simulation Analysis}
\label{sec:simulation}

The Operating point analysis is the following:


The graphs are the following:


\begin{figure}[H] \centering
\includegraphics[width=0.6\linewidth]{vo1.pdf}
\caption{Experimental gain, as a function of frequency.}
\label{fig:acm1}
\end{figure}


\begin{figure}[H] \centering
\includegraphics[width=0.6\linewidth]{vo2.pdf}
\caption{Voltage amplitudes, as a function of frequency.}
\label{fig:acm2}
\end{figure}


\begin{figure}[H] \centering
\includegraphics[width=0.6\linewidth]{vo3.pdf}
\caption{----}
\label{fig:acm}
\end{figure}

\begin{figure}[H] \centering
\includegraphics[width=0.6\linewidth]{vo4.pdf}
\caption{v(out)/v(in)}
\label{fig:acm}
\end{figure}

\subsection{Input and output impedances}
The input impedance measured in ngspice is the following:

\begin{table}[H]
  \centering
  \begin{tabular}{|l|r|}
    \hline    
    {\bf Name} & {\bf Value } \\ \hline
    \input{../sim/z_tab}
  \end{tabular}
  \caption{Input Impedance, from ngpsice.}
  \label{tab:op}
\end{table}

\subsection{Gain}
As we can see by the figures (\ref{fig:forced1} and \ref{fig:acm1}), the theoretical gain and experimental gain match quite well.

