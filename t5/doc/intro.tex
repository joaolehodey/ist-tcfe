\section{Introduction}
\label{sec:introduction}

In this laboratory assignment we seek to build a bandpass filter using an OP-AMP. Particularly we seek to maximize our \textbf{merit figure}, $M$, given by:
\begin{equation*}
    M = \frac{1}{Cost(VoltageGainDeviation+CentralFreqDeviation+10^{-6})}
\end{equation*}
where the voltage gain deviation is the absolute value of the difference between the gain at 1000 Hz and 40 dB; and the central frequency deviation is the absolute value of the difference between the central frequency and 1000 Hz. The central frequency, $f_c$ , is given by the geometric mean of the low cut-off frequency and the high cut-off frequency:
\begin{equation*}
    f_c = \sqrt{f_H f_L} 
\end{equation*}

The circuit used was the following:


\begin{figure}[H] \centering
\includegraphics[width=0.9\linewidth]{circuit.pdf}
\caption{Circuito utilizado}
\label{fig:circuit}
\end{figure}

\section{Presential Lab}

In this lab assigment we were also able to implement this circuit in real life, where we able to measured the gain and the cut-off frequencies.
For the circuit configuration, we chose the following components: 

\begin{table}[H]
\begin{tabular}{|l|l|l|l|l|l|}
R1 & R2 & R3 & R4  & C3 & C4                         \\ 
1000K$\Omega$ & 500 $\Omega$ & 1000K$\Omega$ & 500 $\Omega$ & 220nF & 220nF \\ 
\end{tabular}
\end{table}

With these components we were able to get a voltage gain of approximately $Gain = 40$ dbs, and cut-off frequencies of $330\, Hz$ and $2.23\,KHz$, corresponding $f_L$ and $f_H$, respectively. 
Using ngpsice, we simulated the same circuit, were we obtained the following results:

