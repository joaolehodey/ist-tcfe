\section{Theoretical Analysis}
\label{sec:analysis}



The transfer function is defined as the ration between the output and the input. In our case, the output is $v0$ and the input $vs$:
\begin{equation*}
    T(s)=\frac{v0}{vs}
\end{equation*}
after a little algebra, we get to the following expression:
\begin{equation}
    T(s)=\frac{R_{1}C_{1}s}{1+R_{1}C_{1}s} (1 + \frac{R_{3}}{R_4}) (\frac{1}{1+ R_{2} C_{2}s}) 
\end{equation}
where, as usual
\begin{equation*}
    s = j\omega
\end{equation*}


The theoretical cut-off frequencies, $f_L$ and $f_H$, can be calculated by the Short Circuit Time Constants Method. They are given by\footnote{If you want to see the deduction in detail, you may visit the following link: \url{https://ocw.mit.edu/courses/electrical-engineering-and-computer-science/6-012-microelectronic-devices-and-circuits-fall-2009/lecture-notes/MIT6_012F09_lec23.pdf?fbclid=IwAR3ezEOiIWVJ0LyNLNp49EwgcpWSC-_IQFO6wASvf9cKXiGx2_0zBplPnb8}}:
\begin{equation}
    f_L = \frac{1}{R_1C_1}
\end{equation}
\begin{equation}
    f_H = \frac{1}{R_2C_2}
\end{equation}
where $f_H$ is the hight cut-off frequency and $f_L$ is the low cut-off frequency.
Experimentally, the cut off frequencies will be calculated through the following expression:
\begin{equation*}
    f = \frac{V_{max}}{\sqrt{2}}
\end{equation*}
where $f$ can be either $f_H$ or $f_L$.


The Results obtained using octave were the following:

\input{../mat/results_octave.tex}