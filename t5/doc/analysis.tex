\section{Theoretical Analysis}
\label{sec:analysis}

\subsection{Input and output impedances.}
To determine the input and output impedances, we first replace the Op-Amp with its equivalent circuit, as shown in figure (\ref{fig:eq_circuit}). 

\begin{figure}[H] \centering
\includegraphics[width=0.6\linewidth]{Incremental_circuit.pdf}
\caption{Pass-band circuit, with the amp-pop replaced with its equivalent circuit.}
\label{fig:eq_circuit}
\end{figure}

Considering the amp-op configuration is a non-inverting amplifier(and the ideal amp-op model), we get that the output and input impedances, $Z_O$ and $Z_I$,  are $0$ and $\infty$, respectively, and that the gain $A$ is equal to : $(1+\frac{R_3}{R_4})$. Therefore we get the following circuit, in figure (\ref{fig:eq_circuit2})

\begin{figure}[H] \centering
\includegraphics[width=0.6\linewidth]{Incremental__circuit2.pdf}
\caption{Pass-band circuit, with the amp-pop replaced with its equivalent circuit, using the ideal model aproximation.}
\label{fig:eq_circuit2}
\end{figure}


Finally, we can deduce the expressions for the input and output impedances for the circuit, $Z_I$ and $Z_{0ut}$, (as seen by $V_{in}$ and $V_{out}$, respectively).
From the circuit in figure (\ref{fig:eq_circuit2}), we get that ( there is no effect on the first part of the circuit, by $V_{out2}$, therefore it is not required to short-circuit the output):

\begin{equation}
Z_I(\omega) = Z_{C_4} + R_4 = \frac{1}{j\omega C_4 } + R4   
\end{equation}

As for the output impedance, we need to short-circuit the input, hence we get the circuit in figure (\ref{fig:eq_circuit3}), from the $V_{out_2}$ terminals:

\begin{figure}[H] \centering
\includegraphics[width=0.6\linewidth]{Circuito_incremental_equivalente.pdf}
\caption{Equivalent circuit seen by the terminals of $V_{out_2}$, when $V_I = 0$.}
\label{fig:eq_circuit3}

\end{figure}





Therefore we get that:
\begin{equation}
 Z_O(\omega) = R_2 || C_3 = \frac{R_2}{j\omega C_3 R_2 + 1}  
\end{equation}
´
\subsection{Transfer function}
The transfer function is defined as the ration between the output and the input. In our case, the output is $v0$ and the input $vs$:
\begin{equation*}
    T(s)=\frac{v0}{vs}
\end{equation*}
after a little algebra, we get to the following expression:
\begin{equation}
    T(s)=\frac{R_{1}C_{1}s}{1+R_{1}C_{1}s} (1 + \frac{R_{3}}{R_4}) (\frac{1}{1+ R_{2} C_{2}s}) 
\end{equation}
where, as usual
\begin{equation*}
    s = j\omega
\end{equation*}

\subsection{Cut-off frequencies}
The theoretical cut-off frequencies, $f_L$ and $f_H$, can be calculated by the Short Circuit Time Constants Method. They are given by\footnote{If you want to see the deduction in detail, you may visit the following link: \url{https://ocw.mit.edu/courses/electrical-engineering-and-computer-science/6-012-microelectronic-devices-and-circuits-fall-2009/lecture-notes/MIT6_012F09_lec23.pdf?fbclid=IwAR3ezEOiIWVJ0LyNLNp49EwgcpWSC-_IQFO6wASvf9cKXiGx2_0zBplPnb8}}:
\begin{equation}
    f_L = \frac{1}{R_1C_1}
\end{equation}
\begin{equation}
    f_H = \frac{1}{R_2C_2}
\end{equation}
where $f_H$ is the hight cut-off frequency and $f_L$ is the low cut-off frequency.
Experimentally, the cut off frequencies will be calculated through the following expression:
\begin{equation*}
    f = \frac{V_{max}}{\sqrt{2}}
\end{equation*}
where $f$ can be either $f_H$ or $f_L$.


The Results obtained using octave were the following:

\input{../mat/results_octave.tex}
